\section{Lecture 4. Eigenvalues and Eigenvectors}
\begin{multicols}{2}
\subsection{Definition}
\begin{definition}[Eigenvalues and Eigenvector]
    Let $A\in\mathbb{R}^{m\times m}$, if a scalar $\lambda\in\mathbb{C}$ and a nonzero vector $v\in\mathbb{C}^m$ statisfy the equation 
    \[
        Av=\lambda v
    \]
    then $\lambda$ is called the eigenvalue of $A$ and $v$ is called the eigenvector of $A$ associated with $\lambda$.
\end{definition}
$\Longrightarrow$ 
\[
    \underbrace{(A-\lambda I)}_{\text{\rm singular}}v = 0
\]
The equation $P_A(\lambda)\triangleq det(A-\lambda I)=0$ is called the characteristic euqation of $A$. 
$P_A(\lambda)\triangleq det(A-\lambda I)=\prod_{i=1}^m(\lambda-\lambda_i)$ is a polynomial of $\lambda$ with degree $m$.\\

For each $\lambda$

\newpage
\end{multicols}