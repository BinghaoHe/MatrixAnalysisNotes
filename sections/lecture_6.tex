\section{Lecture 6. Positive Semidefinite Matrix}
\setcounter{theorem}{0}
\begin{multicols}{2}
For $A\in\mathbb{C}^{m\times m}$, if
\begin{itemize}
    \item [-] $A$ is symmetric, we denote $A\in S^m$
    \item [-] $A$ is Hermitian, we denote $A\in H^m$
\end{itemize}
\begin{definition}
    We say $A\in H^m$ is positive semidefinite (PSD) if 
    \[
        x^H A x \geq 0 \quad \forall x\in\mathbb{C}^m
    \]
    We say $A\in H^m$ is positive definite (PD) if 
    \[
        x^H A x > 0 \quad \forall x\in\mathbb{C}^m,x\neq 0
    \]
    We say $A\in H^m$ is indefinite if $A$ is not PSD.
\end{definition}
\subsection{Properties of PSD Matrices}
\subsubsection{Propeties}
\textbf{Property 1}: If $A$ is PSD then any principal submatrix of $A$ ($A_I$) is PSD. \\
\begin{proof} \\
    Let $x\in\mathbb{C}^m$ with $x_{i_k}\neq 0$ for all $i_k \notin I$. \\
    Let $(x_I)_k=x_{i_k}$ for $i_k\in I$. \\
    Then
    \[
        x^HAx = x_I^HA_Ix_I \geq 0,\quad \forall x_I\in \mathbb{C}^{|I|}
    \]
\end{proof}
By Property 1, we know \\
1) 
\[
    A = \begin{bmatrix}
        A_{11} & A_{12} \\ A_{21} & A_{22}
    \end{bmatrix}
\]
\[
    A_{11}\in H^{k\times k}, A_{22}\in H^{(m-k)\times (m-k)}
\]
If $A$ is PSD(PD) $\Longrightarrow$ $A_{11}$, $A_{22}$ are PSD(PD).
2) \\
If $A$ is PSD(PD), then $A_{ii}\geq 0(>0)$

\subsubsection{Eigenvalues of PSD matrices}
\begin{theorem}
    $A\in H^m$ is PSD(PD) if and only if $\lambda_1,\lambda_2,...,\lambda_3\geq 0(>0)$
\end{theorem}
\begin{proof} \\
    Consider EVD of $A$ as 
    \[
        A=V\Lambda V^H, \quad \Lambda=diag(\lambda_1,\lambda_2,...,\lambda_m)
    \]
    Consider $x^HAx=x^HV\Lambda V^Hx=z^H\Lambda z$. 

    $A$ is PSD $\Longleftrightarrow$ $x^HAx\geq 0$ $\forall x\in\mathbb{C}^m$ $\Longleftrightarrow$ $z^H\Lambda z\geq 0$ $\forall z\in\mathbb{C}^m$.

    $z^H\Lambda z=\sum_{i=1}^m \lambda_i |z_i|^2 \geq 0$ $\forall z\in\mathbb{C}^m$ $\Longleftrightarrow$ $\lambda_i\geq 0$ $\forall i=1,...,m$
\end{proof}

If $A$ is PSD(PD),
\begin{itemize}
    \item [-] $Tr(A)\geq 0 (>0)$
    \item [-] $det(A) \geq 0 (>0)$
\end{itemize}

If $A$ is PD, then $A$ is invertible.
\[
    A = V\Lambda V^H
\]
\[
    A^{-1} = V\Lambda^{-1} V^H
\]

\textbf{Property 2}: $A\in H^m$ can be factorized as $A=BB^H$ for some matrix $B$ if and only if $A$ is PSD. \\
\begin{proof} \\
    $\Longrightarrow$:
    \[
        x^HAx = x^HBB^Hx = \|B^Hx\|^2 \geq 0 \Longrightarrow A \text{\rm is PSD}
    \]
    $\Longleftarrow$:\\
    If $A$ is PSD, define $\Lambda^{1/2}=diag(\sqrt{\lambda_1},\sqrt{\lambda_2},...,\sqrt{\lambda_m})$
    \[
        \begin{array}{ll}
            A &= V\lambda V^H \\
            &= V\Lambda^{1/2}\Lambda^{1/2}V^H \\
            &= V\Lambda^{1/2}Q^HQ\Lambda^{1/2}V^H
        \end{array}
    \]
    where $Q$ is a Hermitian matrix. \\
    Define $B\triangleq V\Lambda^{1/2}Q^H$. Then, $A=BB^H$.
\end{proof}

\begin{theorem}
    Consider $A\in H^m$, $B\in\mathbb{C}^{m\times n}$ and $C=B^HAB$
    \begin{itemize}
        \item [-] If $A$ is PSD, then $C$ is PSD.
        \item [-] If $A$ is PD, then $C$ is PD if and only if $rank(B)=n$
        \item [-] If $B$ is square and invertible, then $A$ is PSD(PD) if and only if $C$ is PSD(PD)
    \end{itemize}
\end{theorem}
\begin{proof} of 2)\\
    $(\Longrightarrow)$ Show $C$ is PSD $\Longrightarrow$ $rank(B)=n$\\
    Assume $B$ is not full column rank.\\
    Consider $x^HCx=x^HB^HABx$, then there exists $x\neq 0$ s.t. $Bx=0$ $\Longrightarrow$ $x^HCx=0$ (Contradict!)
    $(\Longleftarrow)$ Show $rank(B)=n$ $\Longrightarrow$ Show $C$ is PSD \\
    $rank(B)=n$ $\Longrightarrow$ $Bx=0$ iff $x=0$. So $\forall x\neq 0$ $\Longrightarrow$ $Bx\neq 0$. \\
    $\Longrightarrow$ $x^HCx = x^HB^H A Bx > 0$ as $A$ is PD. So $C$ is PD.
\end{proof}

\textbf{Property 3}: Let $A\in\mathbb{C}^{m\times k}$ and $B\in\mathbb{C}^{k}$ and $B$ has full row rank ($rank(B)=k$). Then 
\[
    Range(A) = Rnage(AB)
\]
\begin{proof} \\
    $Range(B)=\mathbb{C}^k$ as $B$ is full row rank.
    \[
        \begin{array}{ll}
            Range(AB) &= \{ y|y=ABx, x\in\mathbb{C}^n \} \\
            &= \{ y | y=Az, \underbrace{z=Bx,x\in\mathbb{C}^n}_{z\in Range(B)=\mathbb{C}^k} \} \\
            &= \{ y | y=Az, z\in\mathbb{C}^k \} \\
            &= Range(A)
        \end{array}
    \]
\end{proof}

\textbf{Property 4}: Let $B\in\mathbb{C}^{m\times k}$ and $C\in\mathbb{C}^{m\times k}$ be two matrices with full column rank. Then
\[
    BB^H=CC^H
\]
if and only if $C=BQ$ for some unitary $Q$.
\begin{proof} \\
    ($\Longleftarrow$): obvious. \\
    ($\Longrightarrow$): $(C^H)^{+}\triangleq C(C^HC)^{-1}$. Since $rank(C^HC)=rank(C)$, $C^HC$ is invertible $^\dagger$. \\
    $^\dagger$: As $C$ has full column rank, then $Cx=0$ $\Longleftrightarrow$ $x=0$. 
    $Cx=0 \Longrightarrow C^HCx=0 \Longrightarrow x^HC^HCx=0 \Longleftrightarrow \|Cx\|^2=0 \Longrightarrow x=0$. Therefore, $C^HCx=0\Longrightarrow x=0$ and hence $rank(C^HC)=rank(C)$.

    \[
        \begin{array}{ll}
            BB^H(C^H)^+ &= CC^H(C^H)^{+} \\
            &= CC^HC(C^HC)^{-1} \\
            &= C
        \end{array}
    \]
    $\Longrightarrow$ 
    \[
        \begin{array}{ll}
            C &= BB^H(C^H)^+ \\
            &= B\underbrace{B^HC(C^HC)^{-1}}_{\triangleq Q (\text{\rm square})}
        \end{array}
    \]
    \[
        \begin{array}{ll}
            Q^HQ &= (C^HC)^{-H}C^HBB^HC(C^HC)^{-1} \\
            &= \underbrace{(C^HC)^{-H}C^HC}_{=I}\underbrace{C^HC(C^HC)^{-1}}_{=I} \\
            &= I
        \end{array}
    \]
    Therefore, $Q$ is unitary.
\end{proof}

\subsection{Matrix Inequality}
For $A,B\in H^{m\times m}$, we write $A\succeq (\succ) B$ if $A-B$ is a PSD(PD) matrix.\\
So $A\succeq 0$ if $A$ is PSD.\\
\textbf{Basic Properties}:
\begin{itemize}
    \item [-] If $A \succeq 0$, then $\alpha A \succeq 0$ $\forall \alpha \geq 0$
    \item [-] If $A \succeq 0$, $A\succeq 0$, then $\alpha A+B \succeq 0$
    \item [-] If $A \succeq B$, $B\succeq C$, then $\alpha A \succeq C$
    \item [-] If $A \nsucceq B$, it doesn't imply $B\succ A$
\end{itemize}

Let $\lambda_1(A)\geq \lambda_2(A) \geq \cdots \geq \lambda_m(A)$ for $A\in H^{m\times m}$, then
\begin{enumerate}
    \item $A \succeq I$ $\Longleftrightarrow$ $\lambda_i(A)\geq 1$ $\forall i=1,...,m$
    \item $I \succeq A$ $\Longleftrightarrow$ $\lambda_i(A)\leq 1$ $\forall i=1,...,m$
    \item Suppose $A,B\succ 0$, then $A \succeq B$ $\Longleftrightarrow$ $B^{-1} \succeq A^{-1}$
    \item If $A\succeq B$, then $\lambda_i(A) \geq \lambda_i(B)$ $\forall i=1,...,m$
\end{enumerate}
\begin{proof}\\
    1) \\
    $A\succeq I$ $\Longleftrightarrow$ $A-I\succ 0$ $\Longleftrightarrow$ $V\Lambda V^H -I\succeq 0$ $\Longleftrightarrow$ 
    $V(\Lambda - I) V^H \succeq 0$ $\Longleftrightarrow$ $\Lambda -I \succeq 0$ $\Longleftrightarrow$ $\lambda_i(A)\geq 1$ $\forall i=1,...,m$

    2) similar to 1)

    3) \\
    $\begin{array}{ll}
        A \succeq B & \Longleftrightarrow A-B\succeq 0 \\
                    & \Longleftrightarrow A^{\frac{1}{2}}A^{\frac{1}{2}} - B \succeq 0 \\
                    & \Longleftrightarrow A^{\frac{1}{2}}(I-A^{-\frac{1}{2}}BA^{-\frac{1}{2}})A^{\frac{1}{2}} \succeq 0 \\
                    & \Longleftrightarrow I-A^{-\frac{1}{2}}BA^{-\frac{1}{2}} \succeq 0 \\
                    & \Longleftrightarrow \lambda_i(A^{-\frac{1}{2}}BA^{-\frac{1}{2}} ) \leq 1 \quad \forall i = 1,...,m \\
                    & \Longleftrightarrow \frac{1}{\lambda_i(A^{\frac{1}{2}}B^{-1}A^{\frac{1}{2}} )} \leq 1 \quad \forall i = 1,...,m ^\dagger\\
                    & \Longleftrightarrow \lambda_i(A^{\frac{1}{2}}B^{-1}A^{\frac{1}{2}} ) \geq 1 \quad \forall i = 1,...,m \\
                    & \Longleftrightarrow A^{\frac{1}{2}}B^{-1}A^{\frac{1}{2}} - I \succeq 0 \\
                    & \Longleftrightarrow A^{-\frac{1}{2}}(I-A^{\frac{1}{2}}B^{-1}A^{\frac{1}{2}})A^{-\frac{1}{2}} \succeq 0 \\
                    & \Longleftrightarrow B^{-1} - A^{-1} \succeq 0 \\
                    & \Longleftrightarrow B^{-1} \succeq A^{-1}
    \end{array}$ \\
    $^\dagger$: $\lambda_i(A^{-1})=\frac{1}{\lambda_i(A)}$
    
    4) \\
    $\begin{array}{ll}
        \lambda_i(A)    &=      \lambda_i(A-B+B) \\
                        &\geq   \lambda_i(B) + \underbrace{\lambda_m(A-B)}_{\geq 0 \text{ \rm since }A-B\succeq 0} \\
                        &\geq   \lambda_i(B) \quad \forall i=1,...,m
    \end{array}$
\end{proof}
From 4), if $A\succeq B$ $\Longrightarrow$ $Tr(A)\geq Tr(B)$, if additionally $A,B \succeq 0$ $\Longrightarrow$ $det(A) \geq det(B)$, and if $A,B \succ 0$ and $A\succeq B$ then $Tr(B^{-1})\geq Tr(A^{-1})$

\begin{definition} [Schur Complement]
    Consider a matrix $X\in H^{(m+n)\times (m+n)}$ to be partitioned 
    \[
        X = \begin{bmatrix}
            A & B \\ B^H C
        \end{bmatrix}
    \]
    where $A\in H^{m\times m}$, $B\in \mathbb{C}^{m\times n}$, $C\in H^{n\times n}\succ 0$. The term 
    \[
        S\triangleq A - BC^{-1}B^H \in H^{m\times m}
    \]
    is called \emph{Schur Complement} of $X$.\\
\end{definition}

Then $X\succeq (\succ) 0$ $\Longleftrightarrow$ $S\succeq (\succ) 0$
\begin{proof} \\
Let $Y=\begin{bmatrix}
    I_m & 0 \\ -C^{-1}B^H & I_n
\end{bmatrix} \in \mathbb{C}^{(m+n)\times (m+n)}$. \\
Then $Y$ is a lower triangular matrix and $det(Y)=1$ $\Longrightarrow$ $Y$ is invertible.\\
Then
\[
    Y^HXY =\begin{bmatrix}
        A - BC^{-1}B^H & 0 \\ 0 & C
    \end{bmatrix}
\]  
$Y^HXY \succeq (\succ) 0$ $\Longleftrightarrow$ $x\succeq (\succ) 0$ $\Longrightarrow$ $X\succeq(\succ) 0$ $\Longleftrightarrow$ $X\succeq(\succ) 0$
\end{proof}

\[
    \begin{array}{ll}
        det(YX^HY) &= det(X) \\
        &= det(S)det(C)
    \end{array}
\]

\newpage
\end{multicols}